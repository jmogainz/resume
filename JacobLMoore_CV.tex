\documentclass[10pt, letterpaper]{article}

% Packages:
\usepackage[
    ignoreheadfoot, 
    top=2 cm, 
    bottom=2 cm,
    left=2 cm,
    right=2 cm,
    footskip=1.0 cm,
]{geometry}
\usepackage{titlesec}
\usepackage{tabularx}
\usepackage{array}
\usepackage[dvipsnames]{xcolor}
\definecolor{primaryColor}{RGB}{0, 0, 0}
\usepackage{enumitem}
\usepackage{fontawesome5}
\usepackage{amsmath}
\usepackage[
    pdftitle={Jacob Lane Moore's CV},
    pdfauthor={Jacob Lane Moore},
    pdfcreator={LaTeX with RenderCV},
    colorlinks=true,
    urlcolor=primaryColor
]{hyperref}
\usepackage[pscoord]{eso-pic}
\usepackage{calc}
\usepackage{bookmark}
\usepackage{lastpage}
\usepackage{changepage}
\usepackage{paracol}
\usepackage{ifthen}
\usepackage{needspace}
\usepackage{iftex}

% Ensure PDF is machine-readable:
\ifPDFTeX
    \input{glyphtounicode}
    \pdfgentounicode=1
    \usepackage[T1]{fontenc}
    \usepackage[utf8]{inputenc}
    \usepackage{lmodern}
\fi

\usepackage{times}

% Settings:
\raggedright
\AtBeginEnvironment{adjustwidth}{\partopsep0pt}
\pagestyle{empty}
\setcounter{secnumdepth}{0}
\setlength{\parindent}{0pt}
\setlength{\topskip}{0pt}
\setlength{\columnsep}{0.15cm}
\pagenumbering{gobble}

\titleformat{\section}{\needspace{4\baselineskip}\bfseries\large}{}{0pt}{}[\vspace{1pt}\titlerule]

\titlespacing{\section}{
    -1pt
}{
    0.3 cm
}{
    0.2 cm
}

\renewcommand\labelitemi{$\vcenter{\hbox{\small$\bullet$}}$}

\newenvironment{highlights}{
    \begin{itemize}[
        topsep=0.10 cm,
        parsep=0.10 cm,
        partopsep=0pt,
        itemsep=0pt,
        leftmargin=0.4 cm + 10pt
    ]
}{
    \end{itemize}
}

\newenvironment{onecolentry}{
    \begin{adjustwidth}{0 cm + 0.00001 cm}{0 cm + 0.00001 cm}
}{
    \end{adjustwidth}
}

% Environment to place content on left and date on the right:
\newenvironment{twocolentry}[1]{
    \onecolentry
    \def\secondColumn{#1}
    \setcolumnwidth{\fill,4.5cm}
    \begin{paracol}{2}
}{
    \switchcolumn \raggedleft \secondColumn
    \end{paracol}
    \endonecolentry
}

\newcommand{\sep}{\(\ \diamond\ \)}

\begin{document}

% Header Section
\begin{center}

    {\color{primaryColor}\rule{\linewidth}{0.5pt}}\\[0.5em]

    {\fontsize{25pt}{25pt}\selectfont \textbf{\scshape JACOB LANE MOORE}}

    {\color{primaryColor}\rule{\linewidth}{0.5pt}}\\[-0.8em]
    {\color{primaryColor}\rule{\linewidth}{2.5pt}}

    \vspace{0.5em}
    Huntsville, AL 35806 \sep \href{mailto:jlmoors001@gmail.com}{jlmoors001@gmail.com} \sep (256) 701-3403

\end{center}


\section{\scshape Education}

\begin{twocolentry}{Expected 2027}
    \textbf{MS, Georgia Institute of Technology, Computer Science, Computing Systems}
\end{twocolentry}

\vspace{0.10 cm}
\begin{onecolentry}
\begin{highlights}
    \item GPA: 4.0/4.0
    \item Coursework: Hypervisors, Distributed Systems, Software Architecture, Compilers
\end{highlights}
\end{onecolentry}

\vspace{0.3 cm}

\begin{twocolentry}{Graduated}
    \textbf{BS, Mississippi State University, Computer Engineering}
\end{twocolentry}

\vspace{0.10 cm}
\begin{onecolentry}
\begin{highlights}
    \item GPA: 4.0/4.0
    \item Coursework: Embedded Systems, Systems Programming, Computer Architecture, Operating Systems
\end{highlights}
\end{onecolentry}

\section{\scshape About Me}

\begin{onecolentry}
\begin{tabular}{@{}r l@{}}
Languages & C++/C, Python \\
Development & Linux, Neovim, VSCode, Docker, CMake, Make, Zsh/Bash, Git \\
ML/DL & PyTorch, TensorFlow, RLlib, Scikit-Learn, Pandas, Grafana, Matplotlib \\
Certs & Neural Networks and Deep Learning (DeepLearning.AI, Andrew Ng), CompTIA Security+ \\
\end{tabular}
\end{onecolentry}


\section{\scshape Experience}

\begin{twocolentry}{05/2023 – Current}
    \textbf{Software Design Engineer, Torch Technologies – Huntsville, AL}
\end{twocolentry}

\vspace{0.10 cm}
\begin{onecolentry}
    \begin{highlights}
        \item Developed modular and configurable C++ libraries, utilizing modern abstraction techniques, within a logically clocked, microservices-driven AI training pipeline, ensuring deterministic synchronization of actions, observations, and rewards between a reinforcement learning algorithm and a high-fidelity simulation environment.
        \item Designed and extended a plugin-based architecture for reinforcement learning components—actions, observations, and rewards—leveraging dynamic linking and polymorphism to facilitate rapid iterative experimentation.
        \item Implemented a domain-specific query language using lexer/parser generation tools, enabling users to define complex observation compositions through simple configuration files, thereby reducing boilerplate and enhancing flexibility.
        \item Integrated efficient serialization protocols (e.g., FlatBuffers) and custom UDP-based messaging to enhance inter-service communication, while supporting asynchronous, event-driven pub/sub mechanisms for responsive and scalable distributed systems.
        \item Developed a versatile C++ data model that dynamically loads, transforms, and batches datasets for a variety of deep learning models, with specialized support for state-of-the-art transformers, and built a custom ONNX runtime abstraction to enable efficient model inference.
        \item Engineered a service allocation framework to enable parallel experience gathering for reinforcement learning (RL) training by dynamically isolating groups of services for use by vectorized environments, utilizing docker networks and containerization to maximize hardware utilization and system throughput for large-scale RL experiments.
        \item Developed and maintained a versatile multi-threaded data recording utility library abstracted for seamless integration into any microservice, supporting asynchronous, time-driven and synchronous, event-driven recording modes in logically-clocked distributed systems to capture serialized data for debugging, performance profiling, and offline analysis.
        \item Improved build and deployment pipelines by maintaining complex CMake-based build systems, integrating numerous third-party libraries, and supporting containerized and Linux-based environments, accelerating development cycles and ensuring reliable, efficient builds across diverse platforms.
        \item Contributed bug fixes and enhancements to a division-wide C++ microservices toolbox, strengthening the reliability, maintainability, and performance of shared distributed infrastructure leveraged across multiple projects.
        \item Implemented custom C++ plugins within AFSIM to enable augmented behaviors such as the simulation of alternative/unconstrained sensor capabilities during legitimate simulation runs without impacting baseline simulation state and fidelity, thus expanding analytical depth for AI-driven training and experimentation.
    \end{highlights}
\end{onecolentry}

\vspace{0.4cm}

\begin{twocolentry}{05/2020 – 08/2022 (Summers/Winters)}
    \textbf{Software Engineering Intern, Torch Technologies – Huntsville, AL}
\end{twocolentry}

\vspace{0.10 cm}
\begin{onecolentry}
    \begin{highlights}
        \item Developed Python/C++ data generation/processing pipelines to run, record, and analyze thousands of simulation scenarios, applying machine learning techniques (Gradient Boosting, KNN, DNN) to train predictive classification models for real-time threat assessment and decision-making.
        \item Created a Python-based GUI tool for interactive manipulation of high-resolution elevation and imagery data, automating the segmentation and export of complex terrain tiles into Unreal Engine visualization projects.
        \item Extended Unreal Engine visualization C++ front-end with additional DIS packet handling, enabling real-time visualization of projectiles, detonations, and other scenario events in distributed simulations.
    \end{highlights}
\end{onecolentry}

\vspace{0.4 cm}

\begin{twocolentry}{09/2020 – 05/2022}
    \textbf{Undergraduate Research Assistant, Mississippi State University CSE Department}
\end{twocolentry}

\vspace{0.10 cm}
\begin{onecolentry}
    \begin{highlights}
        \item Strengthened machine learning foundations by researching and applying advanced techniques (e.g., recurrent neural networks, GANs, transfer learning) to predictive maintenance tasks, building a solid theoretical and practical base for subsequent AI projects.
        \item Engineered comprehensive data acquisition pipelines that interfaced directly with specialized sensing hardware, converting raw fiber optic signals into actionable training datasets for neural network modeling.
        \item Explored innovative anomaly detection and time-series clustering algorithms, refining model selection and leveraging deep learning architectures (e.g., LSTMs) to enhance fault prediction accuracy and robustness.
    \end{highlights}
\end{onecolentry}


\section{\scshape Skills}

\begin{onecolentry}
    \begin{highlights}
        \item Core Software Architecture \& Design
        \begin{itemize}
            \item Object-Oriented Design, Design Patterns (GoF), UML Diagrams
            \item Modular, Plugin-based, \& Layered Architectures
            \item Strong understanding of encapsulation, abstraction, and cohesion principles
        \end{itemize}

        \item Operating Systems \& Systems Programming
        \begin{itemize}
            \item Proficient in low-level systems programming (C, C++), including manual memory management and concurrency control
            \item Experienced with POSIX APIs for process/thread management and inter-process communication (IPC)
            \item Expertise in Linux systems programming, debugging, and cross-platform builds
            \item Build Systems (CMake, Make) and Shell/Python Scripting for automation
            \item Virtualization \& Containerization (KVM/QEMU, Docker)
            \item Performance Optimization (Profiling, Cache Utilization, Data Locality)
            \item Secure Coding Practices (CompTIA Security+), Thread Safety
        \end{itemize}

        \item Distributed \& Parallel Computing
        \begin{itemize}
            \item Microservices Architectures with synchronous (gRPC, custom UDP protocols) and asynchronous (RabbitMQ, pub/sub) messaging
            \item Network Programming (TCP/UDP sockets) with async for I/O-bound tasks
            \item Logical-Time Synchronization for distributed systems
            \item Parallelization Techniques (OpenMP, MPI) and Performance Tuning
        \end{itemize}

        \item Machine Learning Integration
        \begin{itemize}
            \item Reinforcement Learning Pipelines (Stable baselines, RLlib)
            \item Deep Learning Frameworks (PyTorch, TensorFlow), ONNX Runtime
        \end{itemize}

        \item Collaboration \& Methodology
        \begin{itemize}
            \item Agile Practices, CI/CD Pipelines, Git Workflows, Coding Standards (C++ Core Guidelines)
            \item Technical Mentorship, Code Reviews, and Documentation Standards (Doxygen)
            \item Continuous Learning, Effective Communication, and Cross-Team Collaboration
        \end{itemize}
    \end{highlights}
\end{onecolentry}


\section{\scshape Projects}

\vspace{0.2 cm}

\begin{onecolentry}
    \textbf{Autonomous RC Car}
    \begin{highlights}
        \item Solely engineered the entire software stack for a low-cost, disposable autonomous RC car designed to sweep for and detonate landmines.
        \item Architected a robust ROS 2 pub/sub service-oriented navigation system fusing raw IMU/GPS data for precise Ackermann steering control via electronic speed controllers.
        \item Implemented a Python-based command controller integrating base station input and RTK correctional data over a mavros radio, ensuring accurate real-time navigation.
        \item Developed all supporting base station software and communication interfaces, seamlessly linking user input to the vehicle’s control system.
    \end{highlights}
\end{onecolentry}

\vspace{0.4 cm}

\begin{onecolentry}
    \textbf{Tiger Language Compiler}
    \begin{highlights}
        \item Designed and implemented a C++ compiler for a functional, ALGOL-like language (Tiger), producing optimized MIPS assembly.
        \item Incorporated semantic analysis and error checking, optimized three-address code IR, and two register allocation strategies:
        \begin{itemize}
            \item Chaitin’s Naive Allocation: A simpler, less optimized approach relying on heuristic allocation.
            \item Briggs Optimization: A comprehensive scheme incorporating control-flow and liveness analysis, as well as instruction cost metrics to minimize load/stores and branches.
        \end{itemize}
        \item Achieved best-in-class performance on benchmark tasks, reducing instruction counts by a factor of 10 compared to peer solutions.
    \end{highlights}
\end{onecolentry}

\vspace{0.4 cm}

\begin{onecolentry}
    \textbf{Before and After Creator}
    \begin{highlights}
        \item Built a locally hosted web application leveraging unsupervised ML and deep learning to match before/after car detailing images.
        \item Automated bulk image processing, generating side-by-side collages for a detailing business, improving workflow efficiency and consistency.
        \item Engineered a custom pairing algorithm to reliably identify corresponding images and streamline final outputs for user download.
    \end{highlights}
\end{onecolentry}

\end{document}
